% RSS 2016 High Level Control of Modular Robots
%%%%%%%%%%%%%%%%%%%%%%%%%%%%%%%%%%%%%%%%%%%%%%%%%%%%%%%%%%%%%%%%%
%%%                    Included packages                 %%%
%%%%%%%%%%%%%%%%%%%%%%%%%%%%%%%%%%%%%%%%%%%%%%%%%%%%%%%%%%%%%%%%%

%%%  Included by IEEE:

\documentclass[conference]{IEEEtran}
\usepackage{times}

% numbers option provides compact numerical references in the text. 
\usepackage[numbers]{natbib}
\usepackage{multicol}
\usepackage[bookmarks=true]{hyperref}

%%%%%%%%%%%%%%%%%%%%%%%%%%%%%%%%%%%%%%%%%%%%%%%%%%%%%%%%%%%%%%%%%
%%%   Additional packages:

\usepackage{color}
\usepackage{mathtools}
\usepackage{amsmath} % assumes amsmath package installed
\usepackage{amssymb}  % assumes amsmath package installed
%\usepackage[final]{pdfpages} % for including pdfs

\usepackage{subcaption}

%%%%%%%%%%%%%%%%%%%%%%%%%%%%%%%%%%%%%%%%%%%%%%%%%%%%%%%%%%%%%%%%%
%%%  Macros:

% For making things invisible during double-blind review. Put "#1" in the
% the braces to make the text appear later.:
\newcommand{\doubleBlind}[1]{} 

% For marking Todos and changes:
\newcommand{\TODO}[1]{ {\bf \textcolor{red}{TODO:} #1 }}
\newcommand{\abj}[1]{\textcolor{blue}{#1}}
\newcommand{\dbj}[1]{\textcolor{blue}{\sout{#1}}}
\newcommand{\cbj}[2]{\textcolor{blue}{\sout{#1}}\textcolor{blue}{~#2}}
\newcommand{\abt}[1]{\textcolor{magenta}{#1}}
% Handy commands
\newcommand{\lt}{{\tt True }}
\newcommand{\lf}{{\tt False }}
\newcommand{\ltnsp}{{\tt True}}
\newcommand{\lfnsp}{{\tt False}}
\newtheorem{definition}{Definition}
\DeclareMathOperator{\F}{\rotatebox[origin=c]{45}{$\Box$}}
\DeclareMathOperator{\X}{\bigcirc}
\DeclareMathOperator{\G}{\Box}
\newcommand{\LTLG}{\G}
\newcommand{\LTLF}{\F}
\newcommand{\LTLX}{\X}
%%%%%%%%%%%%%%%%%%%%%%%%%%%%%%%%%%%%%%%%%%%%%%%%%%%%%%%%%%%%%%%%%

\pdfinfo{
   /Author (Mystery Authors)
   /Title  () %TODO add title
   /CreationDate ()
   /Subject ()
   /Keywords ()
}

%%%%%%%%%%%%%%%%%%%%%%%%%%%%%%%%%%%%%%%%%%%%%%%%%%%%%%%%%%%%%%%%
%%%                     Main document                        %%%
%%%%%%%%%%%%%%%%%%%%%%%%%%%%%%%%%%%%%%%%%%%%%%%%%%%%%%%%%%%%%%%%
\usepackage{graphicx}

\begin{document}


\title{Awesome Autonomous Modular Robots in Unknown Environments}

\author{Mystery Authors}

% \author{\authorblockN{Gangyuan Jing}
% \authorblockA{
% Cornell University\\
% \texttt{gj56@cornell.edu}}
% \and
% \authorblockN{Tarik Tosun}
% \authorblockA{Univ. of Pennsylvania\\
% \texttt{tarikt@grasp.upenn.edu}}
% \and
% \authorblockN{Mark Yim}
% \authorblockA{Univ. of Pennsylvania\\
% \texttt{yim@grasp.upenn.edu}}
% \and
% \authorblockN{Hadas Kress-Gazit}
% \authorblockA{Cornell University\\
% \texttt{hadaskg@cornell.edu}}
% }

\maketitle

\begin{abstract}

We present a fully autonomous modular robot system that can perform complex, high-level tasks in an unknown environment without external sensing or control. The physical robot is composed of modules that support multiple robot configurations. An onboard 3D sensor provides information about the environment, which is used to perform SLAM in the unknown environment and inform exploration and feedback control. No external sensors, pose providers, or beacons are used. A centralized planning algorithm uses the information from the environment and the desired high-level task description to synthesize low-level controllers to perform locomotion, reconfiguration, and special actions. A novel, centralized, self-reconfiguration method is used to change robot configurations when desired. To the authors' knowledge, the proposed work comprises the first modular robot system that intelligently uses reconfiguration to adapt to an \textit{a priori} unknown environment to perform complex tasks with no human aid or external inputs.



\end{abstract}

\IEEEpeerreviewmaketitle

       %     ____      __                 __           __  _
       %    /  _/___  / /__________  ____/ /_  _______/ /_(_)___  ____
       %    / // __ \/ __/ ___/ __ \/ __  / / / / ___/ __/ / __ \/ __ \
       %  _/ // / / / /_/ /  / /_/ / /_/ / /_/ / /__/ /_/ / /_/ / / / /
       % /___/_/ /_/\__/_/   \____/\__,_/\__,_/\___/\__/_/\____/_/ /_/

\section{Introduction} \label{sec:introduction}

Modular self-reconfigurable robot (MSRR) systems are composed of a number of simple repeated robot elements (called \emph{modules}) that connect together to form larger robotic structures. These systems can \emph{self-reconfigure}, changing their shape (\emph{i.e.} the connective structure of the modules) to meet the needs of the task at hand.
In principal, these systems can address a wide variety of tasks by transforming into a wide variety of morphologies. The traditional approach to achieving flexible robots is to build  monolithic systems that are highly capable, but also highly complex (\emph{e.g.} large humanoids).  Self-reconfigurability is an elegant, scalable alternative: since the shape of the robot is not fixed, each individual task can be solved with a morphology that is only as complicated as it needs to be.  

Over the past three decades, dozens of modular robot systems have been built \cite{Yim2007a}. Existing literature provides ample evidence of MSRR systems reconfiguring and assuming interesting morphologies, as well as methods for programming, controlling, and simulating modular robots \cite{Yim2007,Jing2016,Yim1994}. 

These capabilities are impressive, and each represents a significant research accomplishment in its own right. However, in order to truly live up to their promise of flexible capability in the real world, MSRR systems must demonstrate autonomy: moving, navigating, interacting with objects, and self-reconfiguring, all in unknown environments and without external localization or control. To our knowledge, this paper represents the first example of a truly autonomous MSRR system accomplishing tasks in an unknown environment.

Traditional robotics literature provides numerous examples of robots operating autonomously in unknown environments \TODO{jonathan -cite}. Our system goes beyond this existing work because it has the unique ability to recognize and act on situations in which reconfiguration is needed to complete a task.
Through hardware experiments, we demonstrate that autonomous self-reconfiguration allows our system to complete tasks that would have otherwise been impossible.

The remainder of the paper is structured as follows. \TODO{complete paper structure paragraph}


       %     ____       __      __           __   _       __           __
       %    / __ \___  / /___ _/ /____  ____/ /  | |     / /___  _____/ /__
       %   / /_/ / _ \/ / __ `/ __/ _ \/ __  /   | | /| / / __ \/ ___/ //_/
       %  / _, _/  __/ / /_/ / /_/  __/ /_/ /    | |/ |/ / /_/ / /  / ,<
       % /_/ |_|\___/_/\__,_/\__/\___/\__,_/     |__/|__/\____/_/  /_/|_|

\section{Related Work}\label{sec:related-work}
%

\subsection{Mapping and Navigation}\label{mapping-and-navigation}

\TODO{Jonathan, can you add some relevant literature here?}

\subsection{Modular Robots Completing Tasks}
%
MSRR systems have demonstrated the ability to accomplish low-level tasks such as  various modes of locomotion \cite{Yim1994}.
Recently work includes a system which integrate many low-level capabilities of a MSRR system in a design library, and accomplishes high-level user-specified tasks by synthesizing elements of the library into a reactive state-machine \cite{Jing2016}. While this system demonstrates autonomy with respect to task-related decision making, it is designed to operate in a fully known environment with external sensing.

Modular robots have long been regarded as having the potential to make impact in unknown environments (such as search and rescue scenarios), because  self-reconfiguration theoretically gives them the flexibility to respond to whatever they encounter \cite{Yim2007a,Yim2000}.  However, examples of MSRR actually operating in unknown environments are very limited. To our knowledge, no MSRR system has been used for SLAM. We believe our system has more autonomy in an unknown environment than any existing modular robot system, and represents an important step toward the application of MSRR in the real world.

There is work on mapping with swarm robot systems. The Millibot system has demonstrated the ability to map a partially unknown environment when operating as a swarm \cite{Grabowski2000}. The autonomy of the Millibot swarm is limited: a human operator makes all high-level decisions, and is responsible for navigation using a GUI. Certain members of the swarm are designated as ``beacons,'' and have known locations, making the environment only partially unknown.

The swarm-bots are a MSRR system that has been applied in exploration \cite{Dorigo2005} and collective manipulation \cite{Mondada2005} scenarios.  Like the Millibots, exploration is demonstrated in partially unknown environments, with some members of the swarm acting as ``beacons'' with known location.  In a collective manipulation task, the swarm-bots have limited autonomy, with a human operator specifying the location of the manipulation target and the global sequence of manipulation actions. The swarmanoid project (successor to the swarm-bots), moves a step beyond this capability, using a heterogeneous swarm of ground and flying robots (called ``hand-'', ``foot-'', and ``eye-'' bots) to perform exploration and object retrieval tasks in unknown environments \cite{Dorigo2013}. 
%
\subsection{Autonomous Self-Reconfiguration}
\label{autonomous-self-reconfiguration}
%
Autonomous reconfiguration has been demonstrated with several modular robot systems. CKbot, Conro, and MTRAN have all demonstrated the ability to join disconnected clusters of modules together \cite{Yim2007, Rubenstein2004,Murata2006}. In order to align, Conro uses infra-red sensors on the docking faces of the modules, while CKBot and MTRAN use a separate sensor module on each cluster.  In all cases, individual clusters locate and servo towards each other until they are close enough to dock.

While these proof-of-concept experiments demonstrate the ability to reconfigure, these capabilities have not been used as part of a larger system to complete tasks. The experiments do not include any planning or sequencing of multiple reconfiguration actions in order to create a goal structure appropriate for a task.  Additionally, because these are all chain-type modular robots, individual modules are not able to locomote on their own, and mobile clusters of modules are limited to slow crawling gaits.  Consequently, reconfiguration is very time consuming, with a single connection requiring 5-15 minutes.

Other work has focused on reconfiguration planning, but not autonomous reconfiguration.  Paulos et al. present a system in which self-reconfigurable modular boats self-assemble into functional floating structures, such as a bridge \cite{Paulos2015}.  Like the SMORES-EP modules used in this paper, individual boat modules are able to move about the pool, allowing for rapid reconfiguration.  However, external localization is provided by an overhead AprilTag system. 

To our knowledge, this paper represents the first examples of a MSRR system autonomously making the decision to reconfigure in response to its sensed environment, and then actually reconfiguring\ in order to complete its task.  In \cite{Dorigo2013}, hand-bot and foot-bot elements of the swarmanoid system connect and disconnect in order to complete a book-retrieval task, but the decision to take this action is not made autonomously by the robot in response to sensed environment conditions.

\section{System Overview}
% System Overview Figure
\begin{figure}
\begin{center}
\includegraphics[width=0.4\textwidth]{images/overview.png}
\caption{System Overview Flowchart}
\label{fig:overview}
\end{center}
\end{figure}

       %     __  __               __
       %    / / / /___ __________/ /      ______ _________
       %   / /_/ / __ `/ ___/ __  / | /| / / __ `/ ___/ _ \
       %  / __  / /_/ / /  / /_/ /| |/ |/ / /_/ / /  /  __/
       % /_/ /_/\__,_/_/   \__,_/ |__/|__/\__,_/_/   \___/

\section{Hardware} % (fold)
\label{sec:hardware}
%
\subsection{SMORES-EP Modular Robot} \label{sec:smores}
%
Our system is built around the SMORES-EP robot, but could easily be adapted to
work with other hardware platforms.  In this section, we provide a brief
introduction to the technical capabilities of SMORES-EP.

Each module is about the size of an \textit{80mm cube}, and has four actuated DoF - three continuously rotating faces (left, right, and
pan)  and one central hinge (tilt) with a \(180^\circ\) range of motion
(Fig.~\ref{fig:smores-module}). The DoF marked left, right, and tilt  have
 axes of rotation that are parallel and coincident. A single module can use its
left and right wheels to drive around as a two-wheel differential drive robot.
All four faces of the SMORES-EP module have electro-permanent (EP) magnets
that serve as a high-strength, low-energy connector for self-reconfiguration
\cite{tosun2016design}.  Any face of one module can connect to any face of
another.

The magnetic connectors can also attach to objects made of ferromagnetic
materials (such as steel).  By taking advantage of this capability, SMORES-EP
modules can use their magnets to attract, lift, and carry metal objects.
Provided the attachment surface is flat and smooth, the attachment force
between a SMORES-EP face and a strongly ferromagnetic object can be as high as
90N \cite{tosun2016design}.

Each module has an onboard battery, microcontroller, and 802.11b wireless
module to send and receive UDP packets.  In this work, clusters of SMORES
modules were controlled by a central computer running a Python program that
sends wireless commands to control the four DoF and magnets of each module.
Battery life is about one hour (depending on motor, magnet, and radio usage),
and commands to a single module can be received at a rate of about 20hz.
Wireless networking was provided by a standard off-the-shelf  router, with a
range of about 100 feet.

%% SMORES-EP module DoF picture
\begin{figure}   
\begin{center}
\includegraphics[height=1.5in]{images/smores_dof.pdf}
\end{center}
\caption{SMORES-EP module}
\label{fig:smores-module}
\end{figure}
%

\subsection{Sensor Module} % (fold)
\label{sec:sensor_module}
%

In MSRR systems, sensing and processing capabilities of individual modules are severely limited by the size and weight constraints of the module form factor. Therefore, the proposed system uses a special sensor module dedicated to sensing and processing equipment, shown in Figure \ref{fig:sensor-module}. The module has no actuation capabilities and is larger than SMORES-EP modules. It is equipped with a front-facing Xtion Pro Live RGB-D sensor\footnote{TODO: Xtion Ref} which enables the robot to explore and map the environment and recognize objects of interest. A high, downward-facing HD webcam is included to provide a view of the robot itself, which is used for self-reconfiguration. Finally, an UP computing board\footnote{TODO: Up Board Ref} provides high performance I/O and processing capability in a small form factor. The UP Board used in the proposed system has an Intel Atom 1.92 GHz processor, 4 GB memory, and a 64 GB hard drive. It is network connected via 802.11 wifi. A battery provides power to the Up Board with a lifetime of about 1.5 hours.

% Sensor Module Figure
\begin{figure}
\begin{center}
\includegraphics[width=0.4\textwidth]{images/sensor_module.jpg}
\caption{Sensor module. \TODO{get better picture}.}
\label{fig:sensor-module}
\end{center}
\end{figure}

%
% subsection sensor_module (end)
% section hardware (end)
%

\section{Perception and Environment Characterization}

Since the proposed system performs tasks in unknown environments and conditions, a robust suite of perception algorithms is required to inform control and decision-making. The robot must have the ability to explore and build a map of its environment while avoiding obstacles and tracking its pose. The system must be able to recognize objects and regions of interest related to the desired task. Finally, the system must characterize the environment in terms of configuration capabilities. Features in the environment may restrict which robot configurations can viably perform parts of the high-level task, such as retrieving an object or navigating to a waypoint. The system needs to recognize these features to be able to intelligently choose the appropriate robot configuration for performing the task.

To facilitate the use of open source software and enable networking between components, the proposed system is built in a ROS framework.\footnote{http://www.ros.org} Robot pose is provided by a RGB-D SLAM software package called RTAB-MAP.\cite{rtabmap} A 3D map of the environment is incrementally built and stored in an efficient octree-based volumetric map using Octomap.\cite{octomap} Task-related objects and regions of interest in the environment are given distinctive colors. These colors are recognized using color recognition software\footnote{CMVision: http://www.cs.cmu.edu/$\sim$jbruce/cmvision/} and tracked in 3D using depth information from the Xtion sensor.\footnote{Lucas Coelho Figueiredo: https://github.com/lucascoelho91/ballFollower}

The system must explore the \textit{a priori} unknown environment to search for task-related objects and zones. The system performs this exploration in an intelligent manner using a new next best view planner for object exploration created by two of the authors.\footnote{This work has been newly accepted for journal publication in 2017}. The exploration algorithm uses the current volumetric map of the environment and estimates the next reachable sensor viewpoint that will observe the largest volume of undiscovered portions of objects. It also estimates the amount of information (in an entropy-reduction sense) that will be gained from a sensor measurement taken at that viewpoint. To integrate into the proposed system, the exploration algorithm is offered as a service that can be queried by the high-level planner when desired. A volumetric map of the environment and the reachable space of sensor viewpoints in that map must be provided to the algorithm. Note that the reachable space may be dependent on It then computes and returns the next best view for the sensor, which the high-level planner can then set as a navigation waypoint.

\begin{figure}[t]
      \centering
      \begin{subfigure}[t]{0.15\textwidth}
        \includegraphics[width=\textwidth]{images/free.png}
        \label{fig:obja}
        \caption{\textbf{``free'}' environment}
    \end{subfigure}
    \begin{subfigure}[t]{0.15\textwidth}
        \includegraphics[width=\textwidth]{images/ledge.png}
        \label{fig:objb}
        \caption{\textbf{``ledge''} environment}
    \end{subfigure}
        \begin{subfigure}[t]{0.15\textwidth}
        \includegraphics[width=\textwidth]{images/hallway.png}
        \label{fig:objb}
        \caption{\textbf{``hallway''} environment}
    \end{subfigure}
      \caption{Environment characterization types for object retrieval.}
      \label{fig:obj}
   \end{figure}

In order to intelligently choose appropriate configurations when performing tasks, the robot must perceive and characterize its environment into a discrete set of environment types that correspond with configuration capabilities. The proposed system includes a perception component that characterizes the environment for the purpose of retrieving an object. This characterization can then be used by the high-level planner to determine the appropriate configuration and gait for successful object retrieval. The discrete set of environment characterizations are shown in Figure \ref{fig:characters}. Once the object is observed by the robot, the environment is assumed to fall under one of these types. If the object is higher than a threshold level, the environment is characterized as the ``ledge'' environment. This requires a configuration/gait that can reach up to the top of the ledge to retrieve the object. If the object is on the ground, an occupancy grid is created around the object, and all grid cells within a robot radius of obstacles are denoted unreachable. The closest reachable point to the object within $20^o$ of the robot's line of sight to the object is selected. If the distance from this point to the object is over a certain threshold, the environment is characterized as ``hallway''. To retrieve the object in this type of environment, a configuration with a long, thin arm is required to reach between the surrounding obstacles to retrieve the object. If the distance to the object is under the threshold, the environment is characterized as ``free''. This means no special configurations are required to reach the object. Depending on the environment type, the characterization algorithm also determines a navigation waypoint from which the robot can either grasp the object or reconfigure to the appropriate configuration for grasping the object. This information is passed to the high-level planner for use in controller synthesis.

\section{Configuration-Specific Controllers}

\TODO{All that low-level stuff}

\section{Reconfiguration}

One major advantage of modular robot system is the ability to change into different configurations and therefore adapting to different environment.
While there are a lot of research on modular robot reconfiguration algorithms \TODO{cite}, in this work we developed a simple but robust reconfiguration controller for the SMORES modular robot system.


\section{High-Level Planner}
% Automaton
\begin{figure}
\begin{center}
\includegraphics[width=0.4\textwidth]{images/autSimple.png}
\caption{Planner Automaton}
\label{fig:automaton}
\end{center}
\end{figure}

To utilize the sensing and actuation capabilities of the robot as a complete system for accomplishing various tasks, we employ an existing framework called LTLMoP for automatically generating robot controller from user specified high-level instructions using formal method \TODO{cite}.
LTLMoP allows us to use each component of our system as a low-level atomic controller and specify a wide-range of reactive(\TODO{define reactive}) robotic tasks as a set of high-level instructions using these controllers.
We direct readers to \TODO{cite} for more details on LTLMoP.
In this section, we will mainly discuss how we model our system in order to use LTLMoP to automatically generate a high-level controller for our desired robot task.

\subsection{Synthesize a controller}
We first abstract our system and the environment with a set of Boolean propositions.
A system proposition ``pickUp'' is \lt means that the robot is picking up an item and \lf otherwise.
An environment proposition ``pinkObject'' is \lt means that the robot is currently sensing a pink object and \lf otherwise.
Then we can specify the robot task in high-level instructions, such as Structured English, supported by LTLMoP.
Following is an example of a robot task written in Structured English for searching a pink object and picking it up once the robot finds it. \TODO{double check this}

\begin{itemize}
\item do {\bf explore} if and only if you are not sensing {\bf pinkObject} and you are not activating {\bf pickUp}
\item do {\bf driveToObject} if and only if you are not activating {\bf pickUp} and you are sensing {\bf pinkObject}
\item do {\bf pickUp} if and only if you were activating {\bf driveToObject} and you are sensing {\bf arrived}
\end{itemize}

In additions to ``pickUp'' and ``pinkObject'', there are some more propositions. 
``explore'' and ``driveToObject'' are system propositions that each represents a different robot action.
``arrived'' is an environment proposition that is \lt if the robot arrives at the pink object.

Using LTLMoP, we can synthesize a robot controller, if one exists, that satisfies the given robot task. The controller is in the form of a finite state automaton as shown in Fig.~\ref{fig:automaton}. Each state is labeled with the value of all system propositions. We denote a proposition with the value of \lf using a ``!'' prefix. Each transition between two states is labeled with the value of {\tt some} environment propositions. Some states and transitions are omitted for clear presentation.

\subsection{Execute a controller}
Since the synthesized high-level controller is a discrete finite state automaton, we need to implement it continuously in order to control the robot to satisfy the given task. 
Each system proposition is mapped to a low-level control program that commands the robot to perform some behaviors.
For example, the proposition ``pickUp'' is mapped to a program that will be executed when ``pickUp'' is \ltnsp. The program command the robot to perform a predefined behavior that will pick up a small magnetic object in front of the robot.
Each environment propositions is mapped to a low-level sensing program that gathers and processes the information from sensors in order to characterize the environment state.
For example, the propositions ``pinkObject'' is mapped to a sensing program that returns \lt or \lf based on whether the robot can detect a predefined pink object with its camera.

To execute the synthesized high-level controller, we start with the predefined initial state in the finite state automaton.
In each iteration, first we determine the value of each environment propositions by calling each corresponding sensing program.
We then can find the next state in the finite state automaton by taking the transition that matches with the current values of all environment propositions.
At last, we execute the corresponding control program based on the value of each system proposition specified in the next state and continue to the next iteration.
For example, we start in the top state in Fig.~\ref{fig:automaton} and execute the ``explore'' program.
If the robot senses a pink object, the value of ``pinkObject'' is \lt and therefore the next state is the bottom right state. We then stop the ``explore'' program and execute the ``driveToObject'' program.

\section{Experiment Results}

To validate and demonstrate the capabilities of the proposed system, a real-world experiment was run in which a robot composed of SMORES-EP modules completed a complex, high-level task in an unknown environment. Figure \ref{fig:map} shows the environment layout, setup to represent a cluttered office. The high-level task consisted of 5 sub-tasks of 3 different types: explore the environment, find and retrieve 2 objects in the office, and deliver each to a drop-off zone. One object was in the open (a ``free'' environment type), and the other was positioned between two obstacles (a ``hallway'' environment). Obstacles, environment types, and locations of the objects and drop-off zone were unknown to the robot \textit{a priori}.

5 SMORES-EP modules, 1 Sensor Module, and 3 dummy modules were used to compose the robot for performing the task. Objects were grasped magnetically using the electromagnets on SMORES-EP modules. 2 configurations were setup and made available for selection by the high-level planner. Figure {fig:configs} shows the two configurations. A ``car'' configuration was designed to drive with maximal speed and agility and grasp objects that are in ``free'' environments. A ``proboscis'' configuration was setup to have a long arm for reaching between obstacles in a ``hallway'' environment type to grasp objects. However, the locomotion ability of this configuration is limited to forward/backward motion, making it unsuitable for general navigation.

% Map
\begin{figure}
\begin{center}
\includegraphics[width=0.4\textwidth]{images/RSSMap.png}
\caption{Experiment Map}
\label{fig:map}
\end{center}
\end{figure}

\begin{figure}[t]
      \centering
      \begin{subfigure}[t]{0.15\textwidth}
        \includegraphics[width=\textwidth]{images/free.png}
        \label{fig:obja}
        \caption{\textbf{``free'}' environment}
    \end{subfigure}
    \begin{subfigure}[t]{0.15\textwidth}
        \includegraphics[width=\textwidth]{images/ledge.png}
        \label{fig:objb}
        \caption{\textbf{``ledge''} environment}
    \end{subfigure}
        \begin{subfigure}[t]{0.15\textwidth}
        \includegraphics[width=\textwidth]{images/hallway.png}
        \label{fig:objb}
        \caption{\textbf{``hallway''} environment}
    \end{subfigure}
      \caption{Environment characterization types for object retrieval.}
      \label{fig:obj}
   \end{figure}

Fgure \ref{fig:demo} shows snapshots from the experiment run. A video of the entire experiment is also available at \TODO{la link}. The starting location prevented the robot from seeing the objects initially, forcing it to explore the environment to search for them. After a period of exploration the robot discovered the pink object first. The characterization algorithm correctly classified the surrounding environment as a ``hallway'' type, and accordingly the robot navigated in front of the object and reconfigured to the ``proboscis'' configuration. Once this was done, the object was retrieved and pulled out into the open. The robot then dropped the object, reconfigured back to the ``car'' configuration for navigation to the drop-off zone, and again retrieved the object. The robot then navigated to and dropped off the object at the drop-off zone, which was seen during exploration and recorded in the global map built by the SLAM algorithm. Once done, the robot navigated to the green object which was correctly determined to not require any reconfiguration. No further exploration was required since the green object was discovered while the robot was dropping off the pink object. The robot finished the experiment by also delivering the green object to the drop-off zone. Figure \ref{fig:data} shows the final volumetric map and compiled point cloud of the environment built by the robot as it explored the environment and delivered objects. The robot successfully completed all tasks in the experiment in about 26 minutes. Note that the video shows a human reaching into the field to touch the green object once the robot comes into conact with it. This was due to a error on the field resulting in the object becoming stuck between two floor boards, so a human dislodged it so the robot could grasp it normally.

\section{Discussion}

The experiment demonstrated that the proposed system can autonomously perform complex high-level tasks in an unknown environment. It also revealed many issues that make autonomous reactive tasks difficult to achieve in modular systems. This section discusses issues that were observed and solutions that were implemented to overcome them.

Due to limitations of individual modules, robot configurations move extremely slow. In addition, having a robotic system composed of 6 individual robotic components (5 SMORES-EP modules and 1 sensor module) multiplies the chance of a component having a hardware failure. The long runtime combined with multiplied failure points resulted in a high risk of an error occuring during the experiment. Thus, each system component had to be designed with features to make it as robust as possible, and several error recovery features had to be implemented in the system.

Networking was one significant issue. All SMORES-EP modules, 2 external PCs, and the Sensor Module's UP board had to be connected to the same network. The amount of traffic on the network resulted in occasional dropped packets which could cause control failures. To prevent slow networking and dropped packets, a dedicated router was used for the experiment, and all other network routers in the lab on the same frequency were turned off to prevent interference. Commands to modules were sent multiple times to enable recovery from dropped commands.

The resources required to enable the perception capabilities of the system present issues with size and computation speed. As can be seen from Figure {fig:sensor-module}, the sensor module must be significantly larger than a regular module in order to hold the required sensors and computation equipment. This imposes significant limitations on possible configurations and their locomotion abilities, due to the low connection strength and power of modules. To handle this issue, one configuration was used for general locomotion, and the system restricted navigation to flat surfaces since the modules were unable to carry the sensor module over uneven terrain. At the same time, the small computer board used in the sensor module results in low computation power, which causes the sensor processing and perception algorithms to run slowly and have less accuracy. Future autonomous modular systems would benefit significantly from powerful but highly compact, lightweight sensing and processing components that can be carried onboard the robot while imposing minimal size and weight constraints.

As with many autonomous systems, uncertainties such as sensor noise and controller lag create issues with robust behavior. For example, the global environment and robot pose determined by visual SLAM have significant error that builds as time progresses. Also, reconfiguration requires very precise positioning of docking modules for the magnets to engage properly. To increase robustness in these components, checks and reptitive feedback controllers are necessary to recover from errors. Once the robot observes an object and begins navigating towards it, the location of the object is continuously updated as the robot approaches it to correct errors in the global map. Once close to the object, the robot aims at the object again using local sensor data, thereby eliminating errors from the global map. During reconfiguration, a docking module is driven close to the target module and aimed appropriately for docking. Once this is done, the controller performs multiple checks to correct the docking module's final position before driving it into the target module to dock. This lets the robot recover from controller lag or noisy April Tag measurements.

Finally, some simple hardware failures were encountered during experiment runs. These resulted from the fact that all SMORES-EP modules are research prototypes built from scratch. As such, they are more prone to failures such as faulty wiring, microcontroller errors, and position control errors from the custom motor encoders. These issues were addressed by frequent hardware checks and encoder calibrations, and by having several modules on standby to swap for problematic ones. Future experiments can be improved in robustness by improving the hardware quality of modules to bring them closer to the level of a commercial product.

In summary, a novel modular robotic system has been presented that is the first to use perception of an unknown environment to reactively perform complex high-level tasks using intelligent reconfiguration. Copmonents of this system include novel controller synthesis, environment characterization, and self-reconfiguration methods. The system was validated using a physical experiment using a high-level task consisting of a heterogenous set of 5 sub-tasks in an \textit{a priori} unknown environment. The experiment also gave several insights into issues encountered with physical perception-driven modular systems.

\section{Conclusion}

\TODO{Let's wrap this up. Note the last paragraph of Discussion, may allow us to drop this section per Hadas' suggestion.}

\section*{Acknowledgments}
%
This work was funded by NSF grant numbers CNS-1329620 and CNS-1329692.
\TODO{Is this all of them?}


       %     ____       ____
       %    / __ \___  / __/__  ________  ____  ________  _____
       %   / /_/ / _ \/ /_/ _ \/ ___/ _ \/ __ \/ ___/ _ \/ ___/
       %  / _, _/  __/ __/  __/ /  /  __/ / / / /__/  __(__  )
       % /_/ |_|\___/_/  \___/_/   \___/_/ /_/\___/\___/____/

%% Use plainnat to work nicely with natbib. 
\bibliographystyle{plainnat}
\bibliography{references}

\end{document}












